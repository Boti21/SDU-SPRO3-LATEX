\documentclass[../report.tex]{subfiles}
\graphicspath{{\subfix{../image/}}}

\begin{document}

\subsection{Risk Assessment of final product}
Graphic will come soon!
\subsubsection{Failed object avoidance - Countermeasure}
Object avoidance in the final product is/can be done by the means of ultrasonic sensors. 
Failure is reduced by using several sensors together and getting a wider range. Also, by 
using lines the paths will indicated for potential persons in the work area. Moreover, 
algorithms are implemented that strictly control the driving direction – 
which is covered by the ultrasonic sensors always. Only when dropping a pallet there is 
a blind spot in the final product. If there is an object in the pallet space, the product 
would only notice while dropping of the pallet. Hence, there remains a small risk. A counter 
measure could be that the forklift stores the pallet spaces occupation state in non-volatile 
memory or a cloud. This could limit the risk further, as it would not try to put a pallet into 
a filled pallet space. Nonetheless – if a customer is on a pallet space – there is a risk.

\subsubsection{Failed navigation - Countermeasure} 
In the final product the navigation is based on a line-system and IR-sensors. 
When no line is detected anymore the forklift can stop itself. Furthermore, 
the line following has been extensively tested in accordance with the engineering 
method. The forklift went through several testing rounds before being “let” into the warehouse.

\subsubsection{Electrical failure - Countermeasure} 
The engineering method is essential in guaranteeing a safe product. Each of the final assemblies 
components got designed and build by following a cycle of: Idea generation and evaluation, 
calculations/simulations, and testing. The latter two were repeated until the results conformed 
with the initial goals. 
Furthermore, circuits where made on PCB where possible and carefully soldered on veroboards. 
Ultimately, the wiring could be improved.

\subsubsection{Fork breaks - Countermeasure}
The fork as well as the other mechanical parts was designed following the engineering method 
as elaborated above. The final design includes aluminum supports and has bin simulated. 
\subsubsection{Business failure - Countermeasure}
Conduct market research - something which has been done in the scope that the project allows.

\subsubsection{Failed communication with forklift - Countermeasure}
The connection has been also developed in small steps – each time carefully testing the result
 and then extending it. Keeping to the standards (http) is a valuable countermeasure. 
\subsubsection{Cable gets loose - Countermeasure}
Implementing all circuits possible on PCB or Veroboard.


\end{document}
