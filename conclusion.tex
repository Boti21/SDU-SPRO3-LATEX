\documentclass[report.tex]{subfiles}

\begin{document}

All in all this project has been successful. Both in terms of the specified requirements and the learning objectives.

\subsection{Requirements}

The only requirements that were left unfulfilled are related to hardware limitations.
These were object avoidance, a self-made motor driver board and pallet pick up functionality.

\subsubsection{Object avoidance}

The code implementation is done and it has been tested but the ESP32 just did not have enough pins to support the ultrasonic sensors.
This problem arose late in the development process and therefore could not be remedied in time. Potential solutions include having 
another multiplexer but since the ultrasonic sensor readings rely on timing the synchronisation of them would have proven to be 
quite a time consuming challenge.

\subsubsection{Motor driver}

The PCB was designed and manufactured by the university but due to an accident when soldering the copper various planes were bridged
and made the board unusable without being repaired. This happened too late in the process for the team to be able to dedicate 
time for this issue, for example the university would not have been able to make it in time.
Instead two motor drivers were sourced and used. These did also use the same chip as the in-house design. 

\subsubsection{Picking up pallets}

This requirement remained unfulfilled due to a hardware failure. One of the IR sensor array started malfunctioning 
after a pin header was soldered to it. Some of the 8 sensors stopped working completely, other ones changed their output
ranges and altered their output curves rendering the whole array mostly unusable. This problem was encountered after the 
budget has been spent and the development process entered the final stage, therefore the part could not be switched.
This was a problem because the moving direction needs one of the two IR arrays as inputs and when the forklift reverses
the other sensor array has to be used, which was impossible. This made it so that the vehicle could not reliably change directions
and pick up pallets.

\subsection{Learning objectives}

The team management system worked well to distribute tasks and keep track of their completion. With it the team members could 
work on every aspect of the project and hone their skills in all applied disciplines.


The circuit designs allowed the team to use the knowledge acquired during the semester, while mechanically valuable experience
was gained regarding embedding sensors into a model and the usage of high-strength materials such as stainless steel.


The coding was an opportunity to seriously use a version management software with multiple users editing source code, and 
also to use, a more complex, real time operating system alongside a website. These provided valuable experience to all who worked on them.

\subsection{The engineering method}

The team tried its best at adhering to the principles of the engineering method. This can be seen in most every aspect of the project
but there are some mistakes that were discovered just too late and could not be corrected or accounted for with the limited budget and
time-frame available.



\end{document}