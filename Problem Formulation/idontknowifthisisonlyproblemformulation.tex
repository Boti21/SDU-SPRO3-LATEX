\documentclass[11pt]{article}
\begin{document}
    \subsection{raw text}
    % this could be an introduction maybe or somehow part of the problem formulation
    In the third semester our professors assigned us a project where
    the focus was to be the importance of sensors and actuators with
    autonomous vehicles being one example. Following this lead the first
    topic that came to mind was logistics, more precisely forklifts.
    To investigate the applicability of such devices in the 
    aforementioned field our team members conducted market research 
    starting from two separate perspectives, one being more 
    industrial and the other more social.

    % my idea here is that the previous part could be under introduction
    % and this under problem formulation but I want to keep them in the same
    % document for now, for editing and readability purposes
    \subsection{more like a problem formulation???}
    \subsubsection{Forklift related accidents and injuries}
    Forklift related accidents are a common occurrence, both with solid
    stationery objects and moving pedestrians. According to data from "article
    1" there a around 34,000 injuries from forklifts in the US every year. This
    results in a great number of lost work days and extra strain and 
    pressure to make up the deficit.

    Out of 143 incidents, where the collision happened with a solid object
    75 of them are with stationery objects. And 53 of them are collisions 
    between forklifts. Both of these types of incidents are related to 
    human errors, like not paying attention to the surroundings when 
    operating forklifts.

    Out of 322 incidents involving pedestrians around 50\% of them have 
    been caused by a forklift striking a pedestrian by accident.

    The safety of forklifts can be greatly increased with the use for sensors that
    enable the forklifts to avoid obstacles on their own, even with the human 
    error factored in. This is similar to how safety in cars has been greatly
    improved in recent years.

    \subsubsection{Is there a market for self-driving forklifts?}
    Most companies are always competing to maximize the quality of the product 
    while keeping the cost as low as possible. Many companies choose
    to outsource manufacturing to countries where, due to the lower
    average salary, they could further the aforementioned goal.
    This could achieve lower manufacturing prices but sometimes 
    the quality suffered in turn. After learning this some companies
    choose to move manufacturing back to their origin. To stay
    competitive despite the higher costs, automating tasks can be
    a lucrative investment. For example having warehouses that can
    work 24 hours a day can greatly reduce certain costs. All without
    compromising the quality of the products.

\end