\documentclass[11pt]{article}
\begin{document}
% Flavour text as an intro short
% Mindmap 
% Brief description of the mindmap
% Forklift was chosen based on opinions and then further research,
% reference the text from the papers customer also, these will be in the appendix
% conclusion: based on reasearch the problem is solved by an autonomous forklift
% market research of competing products
% conclusion of competing products
% requirements (my stuff here aswell)
    \subsection{Introduction} % make it so the the task/project was given to us from a customer
    % this could be an introduction maybe or somehow part of the problem formulation
    In the third semester our professors assigned us a project where
    the focus was to be the importance of sensors and actuators with
    autonomous vehicles being one example. Following this lead the first
    topic that came to mind was logistics, more precisely forklifts.
    To investigate the applicability of such devices in the 
    aforementioned field our team members conducted market research 
    starting from two separate perspectives, one being more 
    industrial and the other more social.

    % my idea here is that the previous part could be under introduction
    % and this under problem formulation but I want to keep them in the same
    % document for now, for editing and readability purposes
    \subsection{more like a problem formulation???}
    \subsubsection{Forklift related accidents and injuries}
    Forklift related accidents are a common occurrence, both with solid
    stationery objects and moving pedestrians. According to data from "article
    1" there a around 34,000 injuries from forklifts in the US every year. This
    results in a great number of lost work days and extra strain and 
    pressure to make up the deficit.

    Out of 143 incidents, where the collision happened with a solid object
    75 of them are with stationery objects. And 53 of them are collisions 
    between forklifts. Both of these types of incidents are related to 
    human errors, like not paying attention to the surroundings when 
    operating forklifts.

    Out of 322 incidents involving pedestrians around 50\% of them have 
    been caused by a forklift striking a pedestrian by accident.

    The safety of forklifts can be greatly increased with the use for sensors that
    enable the forklifts to avoid obstacles on their own, even with the human 
    error factored in. This is similar to how safety in cars has been greatly
    improved in recent years.

    \subsubsection{Is there a market for self-driving forklifts?}
    Most companies are always competing to maximize the quality of the product 
    while keeping the cost as low as possible. Many companies choose
    to outsource manufacturing to countries where, due to the lower
    average salary, they could further the aforementioned goal.
    This could achieve lower manufacturing prices but sometimes 
    the quality suffered in turn. After learning this some companies
    choose to move manufacturing back to their origin. To stay
    competitive despite the higher costs, automating tasks can be
    a lucrative investment. For example having warehouses that can
    work 24 hours a day can greatly reduce certain costs. All without
    compromising the quality of the products.

    \subsubsection{Real-life applications}
    A research paper from a japanese university revealed 
    that in countries with aging demographics autonomous 
    vehicles such as an autonomous forklift are required 
    to lighten the burden on manpower and manual labor. 
    The authors of this paper detailed an autonomous pallet 
    handling system for forklifts, which is able to handle 
    pallets used for harvesting vegetables with no further 
    human input in an outdoors environment.
    %[Pallet Handling System with an Autonomous Forklift for Outdoor Fields](https://www.proquest.com/docview/2459224227?https://proxy3-bib.sdu.dk/login?url=https://search-proquest-com.proxy3-bib.sdu.dk/scitechpremium?accountid=14211&parentSessionId=ifJwKTKvw59CIPuUwduuKQRIsUDWpPTXKMARKwxGp8M%3D&pq-origsite=summon)
    From a paper about aging populations it is stated that 
    11\% of the world is over 60 years of age and this ratio 
    is expected to rise up to 22\% in 2050.
    From a paper about aging populations it is stated that 
    11\% of the world is over 60 years of age and this ratio 
    is expected to rise up to 22\% in 2050.
    %[The aging population: demographics and the biology of aging](https://ul2nb3kn6e.search.serialssolutions.com/?ctx_ver=Z39.88-2004&ctx_enc=info%3Aofi%2Fenc%3AUTF-8&rfr_id=info%3Asid%2Fsummon.serialssolutions.com&rft_val_fmt=info%3Aofi%2Ffmt%3Akev%3Amtx%3Ajournal&rft.genre=article&rft.atitle=The+aging+population%3A+demographics+and+the+biology+of+aging&rft.jtitle=Periodontology+2000&rft.au=Kanasi%2C+Eleni&rft.au=Ayilavarapu%2C+Srinivas&rft.au=Jones%2C+Judith&rft.date=2016-10-01&rft.pub=Wiley+Subscription+Services%2C+Inc&rft.issn=0906-6713&rft.eissn=1600-0757&rft.volume=72&rft.issue=1&rft.spage=13&rft_id=info:doi/10.1111%2Fprd.12126&rft.externalDBID=BSHEE&rft.externalDocID=A460235991&paramdict=en-UK)

    \subsubsection{Conclusion}
    These four papers clearly indicate that there is indeed 
    a market and an application of highly automated processes
    regarding logistics and forklifts as well. This is how
    our group came to the conclusion that an automated
    forklift would solve a relevant problem all the while 
    fitting in the frame outlined by our professors. 
     % these sentences I'm not sure I phrased correctly so other proof readers are needed

\end{document}