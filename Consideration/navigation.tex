\documentclass[../report.tex]{subfiles}
\graphicspath{{\subfix{../image/}}}
\usepackage{longtable}

\begin{document}

\subsection{Navigation alternatives}
The forklift has to find its way around in the real world. In the end it should
according to the project requirements find and drive to a pallet space. 


\begin{table}[H]
    \centering
        \begin{tabularx}{\linewidth}{L|L|L|L|L}
            \multicolumn{5}{c}{Options for Navigation technology}\\
            \hline
            \textbf{\#} & \textbf{Possible solution} & \textbf{Advantages} & \textbf{Disadvantages} & \textbf{Hardware requirements}\\
            \hline
            1.1& Different colored lines on the floor &
            \begin{itemize}
                \item Different colored lines allow easy target management
                \item Relatively simple algorithms for car control + obstacle avoidance
                \item Cheap in price
                \item Customer can put a custom pattern
                \item Easy to test and debug
            \end{itemize}&
            \begin{itemize}
                \item Calibration of color sensors - shifts with different light levels (Especially an issue at TekExpo)
                \item Not as challenging academically - it has been done before
                \item Not flexible - for long term changes the grid has to be redone
            \end{itemize}
            & Color sensor 80 to 140kr
            \\
            \hline
            1.2& \textbf{Rough GPS location + pallet detection} Finding rough area in which the pallet
            must be and then switch over to pallet
            detection algorithm& \begin{itemize}
                \item Learning about GPS and Gyroscopes
                \item Flexible and adaptable to any warehouse - change/update of grid can be done remotely
            \end{itemize} & \begin{itemize}
                \item GPS indoors is not precise enough
                \item pallet detection as a separate problem
            \end{itemize} 
            & GPS and gyroscope + anything for pallet detection
            \\        
            \hline
            1.3 & Preprogrammed map & & Reliance on some other form of feed back, thus just part of the solution & Microcontroller
            \\
        \end{tabularx}
        \caption{Navigation methods}
        \label{tab:my_label}
    \end{table}
    \begin{table}[H]
        \centering
            \begin{tabularx}{\linewidth}{L|L|L|L|L}
                \multicolumn{5}{c}{Options for Navigation technology}\\ 
            \hline
            1.4 & Bluetooth RTLS BLE & Depending on Bluetooth-version - high precision possible (cm-range) & 
            \begin{itemize}
                \item Deployment of beacons necessary - other groups might use similar technology, which could potentially interfere
                \item Limited range
                \item Human bodies hinder Bluetooth signals to a certain extend
                \item Some sources list just an accuracy of up to a meter - in relation to the forklifts targeted size it would be to imprecise.
                \item Often coupled with other solutions - for example GPS
            \end{itemize}
            & Beacons or microcontrollers (3 pieces) with beacon capabilities - far out of the buget
            \\  
                \end{tabularx}
                \caption{Navigation methods}
                \label{tab:my_label}
            \end{table}     
            \begin{table}[H]
                \centering
                \begin{tabularx}{\linewidth}{L|L|L|L|L}
                    \multicolumn{5}{c}{Options for Navigation technology}\\ 
                    
            \hline
            1.5 & Sensor Fusion (IMU + GPS) & 
            \begin{itemize}
                \item Higly precise and self-correcting 
                \item No range limitations
                \item High Learning-curve (Kalman-Filter + implementation of highly advanced mathematical algorithms in C code, possibly getting experience with Matlab-Codegen)
                \item High flexibility - any point in space could be a target (or avoided)
                \item Good in combination with preprogrammed map
                \item Suitable for fast applications - including drones
            \end{itemize}
            & 
            \begin{itemize}
                \item Pricy navigation system could cost 150-350kr
                \item Very complex implementation
                \item Reliance on several sensors - if one breaks, nothing will work
                \item Possible calibration time
                \item Harder to debug (assumption)
                \item algorithm generated by Matlab may stress the computational abilities of the selected MCU/MPU
                \item No concrete examples on the internet that include GPS in the fusion
                \item \textbf{Large scale focus in the project that is not given by the task of this semester, which is analog sensors}
            \end{itemize}
            & IMU-unit, gyroscope, Matlab toolboxes
               
              
        \end{tabularx}
    \caption{Navigation methods}
    \label{tab:my_label}
\end{table}

Other researched options include using the ESP32 UWB and WPS - both methods have
a similar approach as the Bluetooth-method.

However, it was decided to go with the lines on the floor. As this comes with the 
most advantages that fit to the system and is also a applied in industry warehouse robots.
Additionally, it also allows to focus on (analog) sensors, which is a goal of this project.
    

\end{document}
