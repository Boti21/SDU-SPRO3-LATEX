\documentclass[../report.tex]{subfiles}
\graphicspath{{\subfix{../image/}}}

\begin{document}
\subsection{Processing}
    When considering what processing power to use in the project where embedded
    system is wanted, there are to main candidates the microcontroller or the
    microprocessor. There are different aspect of each processing power that
    can be useful depending on what the project requires. 

    There are some key difference between MCU and MPU:
    \begin{itemize}
        \item Lower Power Consumption (MCU)
        \item Cheaper to Manufacture (MCU)
        \item Better Suited for Highly Intensive Work (MPU)
        \item Possibility to Utilize Cores (MPU)
    \end{itemize}

    Both options are well-suited for running real-time systems where the timing
    of actions is essential.

    For this particular project, the benefits of a Microcontroller Unit (MCU)
    do not play a large role. When comparing the power consumption of the
    Microprocessor Unit (MPU) next to the motors, the MPU uses minimal power in
    comparison. The forklift consists of expensive parts, and most customers
    won't mind the small increase in price for the extra functionality,
    especially when they are already investing a significant amount.
    
    The critical aspect is the workload the processor can handle when running a
    real-time system. This means that no compromises in safety or functionality
    are required for the machine to operate effectively.

    \subsubsection{Choosing the right MPU}
    It was decide that the software should be designed with real-time system i
    mind, because of the important of action taking place at the right time.
    There are a wide variate of option available:
    \begin{itemize}
        \item STM32
        \item RaspberryPi
        \item ESP32
        \item BeagleBone
    \end{itemize} 
    One of these options stands out—the ESP32. It is the most cost-effective
    MPU available on the list while still offering the required features for
    smooth real-time system operations. The ESP32 comes with a single processor
    with two cores, distinguishing itself from some of the more expensive
    options that have two processors with one core each. Although the
    two-processor approach allows for a smoother experience, it comes with
    increased programming complexity.

    For the budget prototype, the ESP32 will do the job effectively. However,
    for a potential continuation of the prototype, it should be considered
    whether the STM32 might be a better option, given its potential advantages.
\end{document}