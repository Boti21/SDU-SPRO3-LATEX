\documentclass[../report.tex]{subfiles}
\graphicspath{{\subfix{../image/}}}
\begin{document}
    \subsection{Technical requirements}
        In order to select certain parts for the project some technical requirements 
        have to be estimated. These include, but are not limited to the mass of the 
        payload, the mass of the vehicle and the desired operating speed of the 
        forklift.

        \subsubsection{Payload} % https://www.cask.com/cans/can-tech-specs/https://baixican.en.made-in-china.com/product/ydUGIOqHCQfi/China-Soda-Can-Packaging-with-Easy-Open-Ends-for-12oz-16oz-250ml-330ml-473ml-500ml-Beverage-Drink.html so far I did not find a paper that would tell me can dimensions
           The group agreed that the payload, for modeling purposes, will be cans of soda.
           \begin{center}
                \begin{enumerate}
                    \item Forklift can lift a pallet of 4 european standard sized cans. Individual can dimensions: Width: 66.1 * 10$^{-3}$ m; Height: 115.2 * 10$^{-3}$ m; Volume: 0.33 L. % As per a chinese can manufacturer 
                    \item Forklift can lift  3.5 kg of mass. This includes 8 cans and a pallet. % 1.3*2 kg for the cans, pallet max 200-300*2 g, rest should be capacity for errors
                    \item Dimensions of the pallet should be 150 * 10$^{-3}$ m by 150 * 10$^{-3}$ m. % 2 cans across with ~20 mm to spare
                    \item Payload can be lifted at least 150 * 10$^{-3}$ m high. % Pallet can be around 30 mm tall
                    \item Two pallets can be stacked on top of each other.
                    \item Lifting speed reaches 10 * 10$^{-3}$ m / s. % Nice and round, seemed reasonable, but just a guess
                \end{enumerate}
            \end{center}

        \subsubsection{Movement}
           \begin{center}
                \begin{enumerate}
                    \item Vehicle can travel at 50 * 10$^{-3}$ m / s.
                    \item Vehicle can turn at least 18 ° / s. % Made up 90 degrees every 5 seconds looked reasonable when I imitated it
                    \item This would be about how accurately the vehicle can follow the path but there is no solution to navigation yet so this could be meaningless with some movement types
                    \item Instead of the previous one we could have something like this: The vehicle can deviate less than 100 * 10$^{-3}$ m left or right every meter traveled.
                    \item Vehicle is able pick up a pallet by inserting the forks into the pallet. % This might have to be rephrased but something like this should be here no? or something about approaching the pallets
                \end{enumerate}
            \end{center}

        \subsubsection{Mass of the vehicle}
        \begin{center}
            \begin{enumerate}
                \item Vehicle should weigh less than 10 kg including the payload. % Major guess
            \end{enumerate}
        \end{center}
\end{document}