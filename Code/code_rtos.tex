\documentclass[../report.tex]{subfiles}
\graphicspath{{\subfix{../image/}}}

\begin{document}
    \subsection{FreeRTOS}
    In the design of an autonomous vehicle, factors such as precise timing,
    responsiveness, and predictability are crucial for the software. If the
    software fails to process inputs from sensors quickly enough, it could lead
    to an accident. These critical aspects can be addressed by utilizing a
    real-time operating system (RTOS) like FreeRTOS. With RTOS task scheduling,
    essential actions such as halting movement when an obstacle blocks the way
    are executed within specified time constraints. This ensures that the
    action of stopping the vehicle is not obstructed by other tasks, thus
    preventing accidents.

    \subsubsection{The Espressif flavor}
    One of the reasons we chose the ESP32 MCU was to leverage its capabilities
    with FreeRTOS. The ESP32 uses a custom flavor of FreeRTOS made by
    Espressif. A key difference for this custom flavor is its support for
    Dual-Core processors. This means that tasks can be distributed across two
    cores, as opposed to the original one core support in FreeRTOS. Another
    advantage is the presence of a HAL, ensuring that MCU changes in the ESP
    ecosystem are durable without the need for significant code maneuvers.

    Throughout this section some FreeRTOS native function are used. These
    functions include:

    \textbf{Tasks:}

    \begin{lstlisting}[language=c,caption={FreeRTOS task creation},label={code:task}]
        // Task to be created.Pointer that will be used as the parameter for the task being 
        void vTaskCode( void * pvParameters )
        {
            for( ;; )
            {
                // Task code goes here.
            }
        }
    \end{lstlisting}

    There are a few things to notice that differ from a function in regular C
    language. Firstly, the function name has a prefix of 'v,' where 'v' stands
    for void, indicating that the function does not return anything. Another
    notable aspect is the use of a 'for' loop. In this context, a task is
    designed to run continuous operations, hence the inclusion of the forever
    loop. It's important to highlight that, unlike native C, the forever loop
    won't block other operations. The 'pv' prefix on the function parameter
    indicates that the return type is a pointer to void. This parameter is used
    to continuously pass values to the task throughout the program

    \textbf{Creating task:}

    \begin{lstlisting}[language=c,caption={Creating a task},label={code:task_create}]
        xTaskCreate( vTaskCode, "NAME", STACK_SIZE, &ucParameterToPass, tskIDLE_PRIORITY, &xHandle );
    \end{lstlisting}

    When creating a task, several parameters need to be fulfilled. These
    include:
    \begin{itemize}
        \item \textbf{vTestCode:} The task function to call, as seen in
        listings \ref{code:task}.
        \item \textbf{NAME:} A given name for the task..
        \item \textbf{STACK\_SIZE:} The size of the task stack, specified in
        bytes.
        \item \textbf{ucParameterToPass:} A pointer to pass parameters to the
        task.
        \item \textbf{tskIDLE\_PRIORITY:} The priority at which tasks are run.
        \item \textbf{xHandle:} The task handle by which the task can be
        referred to.
    \end{itemize}

    The 'x' prefix in the function name indicates that the function returns a
    value. In this case, it signifies the return of the handle, which is used
    to reference to the task later on.

    \subsubsection{Development enviroment}
    A familiar development environment is crucial for ensuring quality and
    productivity in coding. Learning the ins and outs of a new Integrated
    Development Environment (IDE) can be time-consuming. Therefore, it is a
    significant advantage that Espressif has integrated their ESP-IDF into the
    VS Code IDE. VS Code is an IDE familiar to everyone participating in the
    project, and its extensive array of tools further facilitates development.
\end{document}