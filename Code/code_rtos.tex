\documentclass[../report.tex]{subfiles}
\graphicspath{{\subfix{../image/}}}

\begin{document}
    \subsection{FreeRTOS}
    In the design of an autonomous vehicle, factors such as precise timing,
    responsiveness, and predictability are crucial for the software. If the
    software fails to process inputs from sensors quickly enough, it could lead
    to an accident. These critical aspects can be addressed by utilizing a
    real-time operating system (RTOS) like FreeRTOS. With RTOS task scheduling,
    essential actions such as halting movement when an obstacle blocks the way
    are executed within specified time constraints. This ensures that the
    action of stopping the vehicle is not obstructed by other tasks, thus
    preventing accidents.

    \subsubsection{The Espressif flavor}
    One of the reasons we chose the ESP32 MCU was to leverage its capabilities
    with FreeRTOS. The ESP32 uses a custom flavor of FreeRTOS made by
    Espressif. A key difference for this custom flavor is its support for
    Dual-Core processors. This means that tasks can be distributed across two
    cores, as opposed to the original one core support in FreeRTOS. Another
    advantage is the presence of a HAL, ensuring that MCU changes in the ESP
    ecosystem are durable without the need for significant code maneuvers.

    \subsubsection{Development enviroment}
    A familiar development environment is crucial for ensuring quality and
    productivity in coding. Learning the ins and outs of a new Integrated
    Development Environment (IDE) can be time-consuming. Therefore, it is a
    significant advantage that Espressif has integrated their ESP-IDF into the
    VS Code IDE. VS Code is an IDE familiar to everyone participating in the
    project, and its extensive array of tools further facilitates development.
\end{document}