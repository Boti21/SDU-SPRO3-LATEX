\documentclass[../report.tex]{subfiles}
\graphicspath{{\subfix{../image/}}}

\begin{document}
    \subsection{FreeRTOS}
    When designing a autonomous vehicle factors like: precise timing,
    responsiveness, and predictability are crucial for the software.
    If the software dose not process inputs from sensors fast enough it could
    result in a accident. These factors can be forfeited by using a real-time
    operating system like FreeRTOS. With RTOS task scheduling, actions like
    stopping movement when a obstacle is blocking the way, is executed within a
    specified time constraint. Meaning that the action of stopping the vehicle
    is not blocked by another action, thus preventing accidents.

    \subsubsection{The Espressif flavor}
    One of the reasons we choose the ESP32 MCU was to utilize its capability
    with FreeRTOS. The ESP32 liberty uses a custom made flavor made for the
    ESP32 by Espressif. One of the key difference for this flavor is its
    support for Dual-Core processors. Meaning that tasks can be distributed
    across two cores, instead of the original one core support in FreeRTOS.
    Another plus is the presents of a HAL, meaning that MCU change in the
    ESP ecosystem is durable without the big code manuever.
    time-consuming
    \subsubsection{Development enviroment}
    A familiar development environment is key for ensuring quality and
    productivity coding. It is time-consuming to learn the way around a new
    IDE, for this reason it is a big plus that Espressif have made a
    interaction of there ESP-IDF into the VS-Code IDE. VS-Code is a IDE that
    everyone that participate in the project are familiar with and it has a
    wide variate of tools available to help development.
\end{document}