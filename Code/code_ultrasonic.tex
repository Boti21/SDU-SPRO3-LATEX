\documentclass[../report.tex]{subfiles}
\graphicspath{{\subfix{../image/}}}

\begin{document}

\subsection{Ultrasonic Sensors and Object Avoidance}
The ultrasonic sensor uses sonar to determine the distance to an object. 
The sensor reads from 2cm to 400 cm with an accuracy of 0.3cm. To calculate
 the distance from the sensor to the object, a function was created. 
 The function explained is the following:

 \begin{lstlisting}[language=c,caption={The TRIG pin's Code},label={code:ultrasonic}]
    
    // Send a 10us pulse to the sensor's TRIG pin
    gpio_set_level(TRIG_PIN, 1);
    ets_delay_us(10);
    gpio_set_level(TRIG_PIN, 0);

\end{lstlisting}

The sound travels through the air. When it finds an object, 
the waves come back to the sensor. 
The ultrasound receiver (Echo pin) receives the reflected sound (echo):

\begin{lstlisting}[language=c,caption={The ECHO pin's Code},label={code:ultrasonic}]
    
    // While the pin Echo is high
    while (gpio_get_level(ECHO_PIN) == 1){

        //  do nothing

    }

\end{lstlisting}

Taking into account the sound’s velocity in the air (343 m/s) and the 
time that the sound takes to go to the object and come back, the distance
 to the object can be calculated with the following formula:
distance to an object=((speed of the sound)*time)/2
The same in code:

\begin{lstlisting}[language=c,caption={The distance's Code},label={code:ultrasonic}]
    
    // Distance in cm
    distance_obs = (double)time_Echo_High / SOUND_SPEED_IN_US_PER_CM;


\end{lstlisting}

To get the time that the Echo pin is high, the code below was used:

\begin{lstlisting}[language=c,caption={code to get the time that ECHO is high},label={code:ultrasonic}]
    
    // Time that the pin ECHO is high 
    uint32_t time_Echo_High;  // positive
    gptimer_get_elapsed_time(timer, &time_Echo_High);


\end{lstlisting}

The timer was started when the Echo pin was set high, and it was
 stopped when the Echo pin was low.

 \begin{lstlisting}[language=c,caption={The timer's code},label={code:ultrasonic}]
    
    // Start timer 
    gptimer_start(timer);

    // While the pin Echo is high
    while (gpio_get_level(ECHO_PIN) == 1){

        //  do nothing

    }

    // Stop timer
    gptimer_stop(timer);



\end{lstlisting}


\end{document}
