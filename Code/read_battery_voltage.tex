\documentclass[../report.tex]{subfiles}
\graphicspath{{\subfix{../image/}}}

\begin{document}

\subsection{Read battery voltage}
To read the battery voltage, first it is necessary to configure the ADC
 channel of the ESP32 to read the voltage on the pin wanted, in this case the pin A6:

 \begin{lstlisting}[language=c,caption={configure ADC channel},label={code:battery voltage}]
    
    adc_oneshot_config_channel(adc1_handle, ADC1_6 , &config);

\end{lstlisting}

Then, it is needed to read the raw digital value of the voltage on pin A6
 and store it in the variable read battery voltage:

\begin{lstlisting}[language=c,caption={ reading of the raw digital value},label={code:battery voltage}]
    
    adc_oneshot_read(adc1_handle, ADC1_6, &read_battery_voltage); // Dout = read_battery_voltage

\end{lstlisting}

To convert the raw digital value of the voltage to a floating-point value
 representing the actual voltage level, the following line was written:

\begin{lstlisting}[language=c,caption={ convert raw digital value to a floating point},label={code:battery voltage}]
    
    battery_voltage = (float)read_battery_voltage * 2.4 / 4095.0;;


\end{lstlisting}

It was used the formula    Vout= Dout* Vmax/Dmax Where Vout is the output voltage,
 Dout is the raw digital reading result, Vmax is the maximum measurable input analog
  voltage and Dmax is the maximum of the ouput ADC raw digital reading result. 
  In this case, Vmax is 2.4 volts and Dmax is 4095. 


\end{document}
