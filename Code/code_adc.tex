\documentclass[../report.tex]{subfiles}
\graphicspath{{\subfix{../image/}}}

\begin{document}
    \subsection{ADC}
    When programming the ability to use a ADC (Analog to digital converter) to
    read a analog value on the microprocessor is of crucial importance. In our
    project the ADC is use for reading values from our infrared and load
    sensors. On the ESP32 a total of 18 ADC channels are available with a config
    resolution options of 9,10,11 and 12-bits.
    \subsubsection{Configuring}
    It is important to ensure when using RTOS that functions are Thread safe. If
    not it can result in RTOS not being able to handle a task in the desired
    time frame. For this reason the libraries that are
    include in the ESP-IDF environment that are designed with thread safe in
    mind. The particular library used to facilitate ADC is called
    $esp\_adc/adc\_oneshot.h$ that replaced the previous one in version $5.0.4$ of
    ESP-IDF. 

    To configure a ADC unit:
    \begin{lstlisting}[language=c, caption={Configuring ADC unit 1}]
        adc_oneshot_unit_handle_t adc1_handle;
        adc_oneshot_unit_init_cfg_t init_config1 = {
            .unit_id = ADC_UNIT_1,
            .ulp_mode = ADC_ULP_MODE_DISABLE,
        };
        ESP_ERROR_CHECK(adc_oneshot_new_unit(&init_config1, &adc1_handle));
    \end{lstlisting}

    The program utilizes handles to reference the objectets throughout the
    program. When $calling\ adc\_oneshot\_new\_unit$ a new instance is created with
    the specified configuration. In a similar way the channels are configured
    and created to the handle.
    \subsubsection{Reading}
    To read the raw value of the ADC unit 1 channel 0:
    \begin{lstlisting}[language=c, caption={Readning ADC unit 1 channel 0}]
        adc_oneshot_read(adc1_handle, ADC1_0, &adc_value);
    \end{lstlisting}
    This function takes 3 augments: the handle itself, the desired channel to
    read from and where the output should be stored.
    This function can acquire the raw value from any ADC on the given unit as
    long as they are configured.

    To calculate the voltage level:
    \begin{equation}
        V_{out}=D_{out}*\dfrac{V_{max}}{D_{max}}
    \end{equation}
    
    Where:

    $D_{out}$: ADC raw digital reading result

    $V_{max}$: Max measurable analog input voltage 

    $D_{max}$:$2^{bitwidth}$

    \subsubsection{Implementation}
    The Analog-to-Digital Converter (ADC) is employed to convert analog signals
    from both the infrared and load cell sensors. The two infrared sensors
    collectively generate 16 analog outputs, while the load cell sensor
    contributes one. However, the specific ESP32 utilized in this project
    features only 12 available ADC pins on the breakout board. This poses a
    constraint on the number of outputs from the infrared sensors that can be
    accommodated.

    In the context of the infrared sensors, the ADC value varies based on the
    reflective properties of the surface. To facilitate line-following, the
    program assesses the ADC values and executes movement actions based on the
    reflectivity of the surface for each sensor in the array.
\end{document}