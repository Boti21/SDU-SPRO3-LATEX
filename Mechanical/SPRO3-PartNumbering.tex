\documentclass[../report.tex]{subfiles}
\graphicspath{{\subfix{../img/}}}

\begin{document}    
    \subsection{Part numbering}
    Part numbers begin with FL - indicating that they belong to our project (FL = Forklift). 
    The forklift is split into sections. For example, base, fork and top. These are just
    examples and we can adapt them when the time comes and we start designing.
    So far, the number consists out of the following: FL-Xx, where Xx is being replaced by the starting
    letters of the subsection - for instance, Ba for base. This is then followed by D or M, which indicates
    whether it is a drawing or a model. Giving as a formula:
    FL-XxX. This is then followed by the version number -xx. \textbf{The numbering starts with 00}. Leading to the final 
    and \textbf{general formula}:

    \begin{itemize}
        \item \textbf{FL-XxX-xx}
    \end{itemize}

    The general assembly is marked as \textbf{FL-AsX-xx}.

    \subsubsection{Examples}
    Part number: 
    \begin{itemize}
        \item FL-BaM-00 - Project forklift - Base Model - version 0 (meaning original)
        \item FL-FoD-04 - Project forklift - Fork Drawing - version 4
    \end{itemize}

\end{document}