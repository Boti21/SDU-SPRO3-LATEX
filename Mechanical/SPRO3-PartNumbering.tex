\documentclass[../report.tex]{subfiles}
\graphicspath{{\subfix{../img/}}}

\begin{document}    
    \subsection{Part numbering}
    We introduced a part numbering system in order to keep a clean overview about
    which components we create, manufacture are currently working on. Applying this 
    also allows for intuitive reopening of older versions and picking up the project
    from another colleague.

    Part numbers begin with FL - indicating that they belong to our project (FL = Forklift). 
    The forklift is split into sections. For example, base, fork and top. These are just
    examples and we can adapt them when the time comes and we start designing.
    So far, the number consists out of the following: FL-A/P-XxX, where A/P is either A for assembly or P for part. 
    XxX is being replaced by the starting
    letters  of the subsection for assemblies and starting letters of Component for a part - for instance, Ba for base. (Part)
    or DP for Development Platform (Assembly).
    
    Drawings will be exported under the same name and stored in the same folder as the part (for now)
    Giving as a formula:
    FL-A/P-XxX. This is then followed by the version number -xx. \textbf{The numbering starts with 00}. Leading to the final 
    and 
    \textbf{general formula}:

    \begin{itemize}
        \item \textbf{FL-A/P-XxX-xx}
    \end{itemize}

    \subsubsection{Examples}
    Part number: 
    \begin{itemize}
        \item FL-P-Ba-00 - Project forklift - Part - Base - version 0 
        \item FL-A-MF-04 - Project forklift - Assembly - Mast Fork - version 4
    \end{itemize}

\end{document}