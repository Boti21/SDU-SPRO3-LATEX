\documentclass[../report.tex]{subfiles}
\graphicspath{{\subfix{../image/}}}

\begin{document}
\subsection{Manufacturing methods}
    The method by which components are manufactured has a significant influence
    on various aspects of the project development process. If the chosen
    manufacturing method for a prototype is expensive, it can lead to the
    project exceeding its budget. Similarly, if the manufacturing processes are
    too time-consuming, it can result in delays for the project.
\subsubsection{3D prining}
    Employing 3D printing technology to manufacture components for prototype
    projects offers a many advantages. This cutting-edge manufacturing method
    allows for rapid and cost-effective iteration, enabling us to swiftly refine
    and test various design iterations. The flexibility of 3D printing allows the
    production of intricate and complex geometries that might be challenging or
    impossible with traditional manufacturing methods.

    For manufacturing prototypes using 3D printing, the FDM method was chosen.
    In particular, the materials PLA and PETG were used in the manufacturing
    process. Both materials are non-toxic, meaning they are safe to use
    indoors. However, they do produce volatile organic compounds (VOCs) and
    microplastics in the air, so a well-ventilated workspace is recommended.
    Another great benefit is the low cost of the materials. PETG is particularly
    used where durability is important, but it is not used everywhere because
    it is more expensive than PLA. In most cases, PLA can do the job
    effectively.

    The flexible material TPU was utilized to manufacture the outer layer of
    the wheels. TPU is known for its exceptional flexibility and high friction,
    similar to rubber. This makes it well-suited for the production of wheels
    and other flexible components.
\subsubsection{Laser cutting}
    3D printing is excellent for intricate designs with relatively small areas.
    However, as the printing area becomes larger, it can become expensive and
    time-consuming. In such cases, laser cutting emerges as a more favorable
    option. When using laser cutting, the design must be created in 2D. The
    thickness (the third dimension) depends on the chosen cutting material.

    In the prototype, laser cutting has been employed in various areas, with
    one notable example being the pallet. Due to its large area and fewer
    intricate details, the pallet is an ideal component to manufacture using a
    laser cutter.

\subsubsection{Other methods}
    In the initial phase, the base plate for the forklift was laser-cut.
    However, it was quickly discovered that the available materials were not
    durable enough to withstand the load introduced by the fully loaded pallet.
    The decision was made to manufacture the base plate using stainless steel,
    which was readily available on campus. This choice enhanced the durability
    and weight of the base, contributing to the counterbalance force for the
    forklift.
\end{document}