\documentclass[../report.tex]{subfiles}
\graphicspath{{\subfix{../image/}}}

\begin{document}
\maketitle

\subsection{Lifting Development}
Through choosing the lead screw as a lifting solutions we have to find the
the minimum torque our motor has to put out in order drive the screw under full load.
As we got our lifting motor, the JGY370 sourced in the beginning of the semester without 
any of the further calculations, this will also state if it will fit our project requirements.


To find the Torque, which our Motor has to supply we derive from $Torque=Force*Distance$, where the 
distance is the radius of the lead screw and the force is the force of the load divided by the Mechanical 
Advantage gained through the lead screw. $F_{total}=F_{Load}/MA$

\[F_{Load}=Mass*G=3.5kg*9.81\frac{m}{s}=34.34N\]

The Mechanical Advantage is the product of Velocity Ratio and Efficiency dependent on the lead 
angle. $MA=VR*E$ 

\[VR=\frac{Circumference}{Lead}=\frac{(\pi*10)}{1}=15.7\]

Efficiency shows the impact of friction on the system, which can be accounted for as as increase in lead angle of the threads 
as in the ratio between angle and friction angle. A higher friction results therefore equally to adding more load in an ideal 
system. The friction in this case is described as the \cite{friction_coefficient}[static 
friction coefficient of hard steel on hard steel]. 

\[friction_angle= atand(friction_coefficient)= atand(0.2)=\]
\[E=\frac{tand(lead_angle)}{tand(lead_angle+friction_angle)}=\frac{tand(3.64)}{tand()}\]

\end{document}