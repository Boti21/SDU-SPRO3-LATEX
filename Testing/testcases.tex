\documentclass[../report.tex]{subfiles}
%\graphicspath{{\subfix{../img/}}}

\begin{document}
When testing the functionalities it is important to clearly define what the
expected behavour would be before preforming the test.
\subsection{Test cases}
\begin{table}[H]
    \begin{center}
        \begin{tabularx}{\linewidth}{c|L|L|L|c}
            \multicolumn{5}{c}{Stage 1}\\
            \hline
            \textbf{\#} & \textbf{Objective} & \textbf{Steps} & \textbf{Expected result} & \textbf{P/F}\\
            \hline
            1.1&Forward movement&1.Flash stage 1 code 2.Place forklift down on
            the floor 3.Press start&The
            forklift moves forward a arbitrary distance\\\hline
            1.2&Backward movement&---''---&The
            forklift moves backward a arbitrary distance\\\hline
            1.3&Forward right movement&---''---&The
            forklift moves forward and right a arbitrary distance and angle\\\hline
            1.4&Forward left movement&---''---&The
            forklift moves forward and left a arbitrary distance and angle\\\hline
            1.5&Stationary right turn&---''---&The
            forklift turns right a arbitrary angle\\\hline
            1.6&Stationary left turn&---''---&The
            forklift turns left a arbitrary angle\\\hline
            1.7&Read battery voltage&1.Measure the battery voltage with a
            multimeter 2.Flash stage 1 code 3. Read displayed voltage in
            display&The multimeter measurement and display value should be the
            same with a tolerance of 2\%&\\\hline
            1.8&Battery voltage low&---''---&The forklift should not start when
            the start button is pressed. Instead it should prompt the user that
            the battery needs to be charged.
        \end{tabularx}
    \end{center}
\end{table}

\begin{table}[H]
    \begin{center}
        \begin{tabularx}{\linewidth}{c|L|L|L|c}
            \multicolumn{5}{c}{Stage 2}\\
            \hline
            \textbf{\#} & \textbf{Objective} & \textbf{Steps} & \textbf{Expected result} & \textbf{P/F}\\
            \hline
            2.1&Movement following the line path&1.Flash stage 2 code 2.Place the forklift on line 3.Press start&The forklift should
            follow the line path when moving at a arbitrary speed.&\\\hline
            2.2&Detected obstacle stop&---''---4.Place obstacle on path&The forklift
            should stop before hitting the obstacle&\\\hline
            2.3&Detected obstacle start&1.Run test case 2.2 first. 2.Remove obstacle for
            the path&The forklift resumes it movement along the path&\\\hline
        \end{tabularx}
    \end{center}
\end{table}

\begin{table}[H]
    \begin{center}
        \begin{tabularx}{\linewidth}{c|L|L|L|c}
            \multicolumn{5}{c}{Stage 3}\\
            \hline
            \textbf{\#} & \textbf{Objective} & \textbf{Steps} & \textbf{Expected result} & \textbf{P/F}\\
            \hline
            3.1&Move fork up&1.Flash stage 3 code 2.Place the forklift on line 3.Press start&&\\\hline
            3.2&Move fork down&---''---&&\\\hline
            3.3&Move fork to minimum&---''---&&\\\hline
            3.4&Move fork to maximum&---''---&&\\\hline
            3.5&Detected pallet&---''---&The forklift should detected and move
            into position of the pallet&\\\hline
            3.6&Move pallet up&---''---&The forklift should move the forks
            under the pallet and move it up a arbitrary distance&\\\hline
            3.7&Move with pallet&---''---&The forklift lift should move with
            the pallet to the destination point along the path&\\\hline
            3.7&Move pallet down&&The forklift lift should move the pallet down
            to ground level
        \end{tabularx}
    \end{center}
\end{table}

\end{document}